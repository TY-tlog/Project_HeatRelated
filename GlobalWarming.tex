\documentclass{article}

% Language setting
% Replace `english' with e.g. `spanish' to change the document language
\usepackage[english]{babel}

% Set page size and margins
% Replace `letterpaper' with`a4paper' for UK/EU standard size
\usepackage[letterpaper,top=2cm,bottom=2cm,left=3cm,right=3cm,marginparwidth=1.75cm]{geometry}

%\documentclass[final,3p,times]{elsarticle}

\usepackage{amsfonts,amsmath,amsthm,amssymb,nccmath}
\usepackage{graphicx,epsfig,pstricks,pst-node}
\usepackage{lineno}
\usepackage{figsize}
\usepackage[normalem]{ulem}
\usepackage{dblfloatfix} % fix for bottom-placement of figure

\usepackage{kotex}

\graphicspath{{./Figs/}}

\newtheorem{definition}{Definition}
\newtheorem{theorem}{Theorem}
\newtheorem{lemma}{Lemma}
\newtheorem{proposition}{Proposition}
\newtheorem{example}{Example}
\newtheorem{remark}{Remark}
\renewcommand{\thefootnote}{\fnsymbol{footnote}}

\title{Global Warming and Heat-Related Diseases}
\author{Kim Tae Yoon, Bae Da Eun}

\begin{document}
\maketitle

\begin{abstract}

\end{abstract}


\section{Strategies}
\subsection{Data}
1. Population(whole region, part region)\\
2. Temperature(whole region, part region, world depend on time)\\

\subsection{Data Analysis}
1. 세계기후 변화에 대한 그래프 그림.\\
2. 그 위에 날짜 맞춰서 대한민국 전체 기온 변화 그림.\\
3. 그 위에 날짜 맞춰서 지역별 전체 기온 변화 그림.\\
4. 전체 인구 및 남녀 비율 히스토 그램으로 그리기.\\
5. 온열 환자 발생 수(남녀구분없이) 히스토그램으로 그리기.\\
6. 그 위에 남녀 구분 히스토그램 하나 그리기.\\
\subsection{Prediction}

\end{document}